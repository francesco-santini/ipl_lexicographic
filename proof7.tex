\documentclass[a4paper]{elsarticle}

\usepackage{booktabs} % For formal tables
\usepackage{amsmath,amssymb,amsthm} 
\usepackage{todonotes}
\usepackage{amsfonts}
\usepackage{algorithm}
\usepackage[noend]{algpseudocode}

\newtheorem{definition}{Definition}
\newtheorem{example}{Example}
\newtheorem{remark}{Remark}
\newtheorem{lemma}{Lemma}
\newtheorem{proposition}{Proposition}
\newtheorem{theorem}{Theorem}

\newcommand{\monop}{\otimes}
\newcommand{\1}{\mathbf{1}}


\floatstyle{plain}
\newfloat{myalgo}{tbhp}{mya}

\newenvironment{Algorithm}[2][tbh]%
{\begin{myalgo}[#1]
		\centering
		\begin{minipage}{#2}
			\begin{algorithm}[H]}%
			{\end{algorithm}
		\end{minipage}
\end{myalgo}}

\newcommand{\proc}[1]{\textbf{procedure }\texttt{#1}}
\newcommand{\func}[1]{\textbf{function }\texttt{#1}}
\newcommand{\call}[1]{\texttt{#1}}
\newcommand{\tabul}[1]{\hspace*{#1em}\hspace*{#1em}}
\newcommand{\mywhile}{\textbf{while}}
\newcommand{\mydo}{\textbf{do}}
\newcommand{\myuntil}{\textbf{until}}
\newcommand{\myswitch}{\textbf{switch}}
\newcommand{\myif}{\textbf{if}}
\newcommand{\myendif}{\textbf{endif}}
\newcommand{\mythen}{\textbf{then}}
\newcommand{\myelse}{\textbf{else}}
\newcommand{\myfor}{\textbf{for each}}
\newcommand{\myrepeat}{\textbf{repeat}}

\newcommand{\myendfor}{\textbf{endfor}}
\newcommand{\myendwhile}{\textbf{endwhile}}
\newcommand{\myor}{\textbf{or}}
\newcommand{\myand}{\textbf{and}}
\newcommand{\myno}{\textbf{no}}
\newcommand{\myret}{\textbf{return}}
\newcommand{\myendf}{\textbf{endfunction }}
\newcommand{\myendp}{\textbf{endprocedure }}



\def\odiv{{ \ominus\hspace{-8pt}:}\;}
\def\odivsmall{{ \ominus\hspace{-7.5pt}:}\;}
\def\smallodiv{{ \ominus\hspace{-7.45pt}:}\;}


%%%%%%%%%%%%%%%%%%%%%%%%%%%%%%%%%%%%%%%%%%%
%%%%%%%%%%%%%%%%%%%%%%%%%%%%%%%%%%%%%%%%%%%
\begin{document}





\section*{Appendix}
\begin{proposition}\label{prop:lexiSLM}
	Let $\mathcal{A}$ be a finitely distributive SLM (distributive CLM).
	Then so is $Lex^\omega(\mathcal{A})$
	and for any finite subset (any subset) $X \subseteq A^\omega_L$ 
	we have $(\bigvee X)_1 = \bigvee \{ a \mid \langle a \rangle \in X_{\mid 1}\}$
	and $(\bigvee X)_{i+1} = \bigvee \{ a \mid \langle (\bigvee X)_1, \ldots, (\bigvee X)_i, a \rangle \in X_{\mid i+1}\}$.
\end{proposition}
\begin{proof}
	The proof that $\bigvee X$ of the LUB follows the same steps of the one given in Theorem~\ref{theo:lexiSLM}.
	For distributivity we need to prove that for all $s \in Lex(\mathcal{A})$ and 
	$X \subseteq Lex^\omega(\mathcal{A})$ we have $s \otimes^\omega_\mathcal{A} \bigvee X = \bigvee s \otimes^\omega_\mathcal{A} X$, 
	with $s \otimes^\omega_\mathcal{A} X = \{ s \otimes^\omega_\mathcal{A} t \mid t \in X\}$.
	
	We retrace the steps of Proposition~\ref{def:lexBI}, so let $Y = \{ s \otimes^\omega_\mathcal{A} t \mid t \in X\}$.
	By distributivity of $\mathcal{A}$ we have that
	$(s \otimes^\omega_\mathcal{A} \bigvee X)_1 = s_1 \otimes_\mathcal{A} (\bigvee X)_1 = s_1 \otimes_\mathcal{A} \bigvee X_{\mid 1}
	= \bigvee s_1 \otimes X_{\mid 1} =  \bigvee Y_{\mid 1} = (\bigvee Y)_1$ and
	$$Y_2 = \{ s_2 \otimes_\mathcal{A} d \mid (\bigvee Y)_{1} \cdot (s_2 \otimes d) \cdot w \in Y\}
	= \{  s_2 \otimes d \mid (s_1 \otimes_\mathcal{A} (\bigvee X)_{1}) \cdot (s_2 \otimes_\mathcal{A} d) \cdot w \in Y \}$$
	%
	If $s_1$ is collapsing then we are done, since $s_2 = \bot_\mathcal{B}$ and $Y_2 \subseteq \{\bot_{A}\}$,
	thus $s_2 \otimes_\mathcal{A} \bigvee X_1 = 
	\bigvee Y_2 = \bot_{A}$, and consequenty $w = \bot^\omega$.
	%even if either $X_\mathcal{B} \neq \emptyset$ or $Y_\mathcal{B} \neq \emptyset$. 
	Let us assume that $s_2$ in cancellative.
	Note that $(\bigvee X)_1 \not \in X_{\mid 1}$ implies $(\bigvee X)_2 = \bot_{\mathcal A}$ as well as, 
	since $a$ is cancellative, $Y_2 = \emptyset$.
	Thus $(\bigvee X)_2 = \bigvee Y_2 =\bot_\mathcal{B}$, as well as 
	$w = \bot^\omega$, and we are done.
	Finally, let us assume that $(\bigvee X)_1 \in X_{\mid 1}$. Then
	$$Y_2 = \{  s_2 \otimes d \mid (\bigvee X)_1 \in X_{\mid 1} \wedge (s_1 \otimes_\mathcal{A} (\bigvee X)_{1}) \cdot (s_2 \otimes_\mathcal{A} d) \cdot w \in Y \}
	=  $$
	$$\{  s_2 \otimes d \mid (\bigvee X)_1 \in X_{\mid 1} \wedge (s_1 \otimes_\mathcal{A} (\bigvee X)_{1}) \cdot (s_2 \otimes_\mathcal{A} d) \in Y _{\mid 2} \}
	=$$
	$$\{  s_2 \otimes d \mid \langle (\bigvee X)_1, d \rangle \in X _{\mid 2} \}
	= s_2 \otimes \{ d \mid \langle (\bigvee X)_1,  d \rangle \in X _{\mid 2} \}$$
	%
	We finally proceed again by induction and case analysis on $s_2$ and $\bigvee Y_2$.
\end{proof}

\end{document}

