\documentclass[a4paper]{elsarticle}

\usepackage{booktabs} % For formal tables
\usepackage{amsmath,amssymb,amsthm} 
\usepackage{todonotes}
\usepackage{amsfonts}
\usepackage{algorithm}
\usepackage[noend]{algpseudocode}

\newtheorem{definition}{Definition}
\newtheorem{example}{Example}
\newtheorem{remark}{Remark}
\newtheorem{lemma}{Lemma}
\newtheorem{proposition}{Proposition}
\newtheorem{theorem}{Theorem}

\newcommand{\monop}{\otimes}
\newcommand{\1}{\mathbf{1}}


\floatstyle{plain}
\newfloat{myalgo}{tbhp}{mya}

\newenvironment{Algorithm}[2][tbh]%
{\begin{myalgo}[#1]
		\centering
		\begin{minipage}{#2}
			\begin{algorithm}[H]}%
			{\end{algorithm}
		\end{minipage}
\end{myalgo}}

\newcommand{\proc}[1]{\textbf{procedure }\texttt{#1}}
\newcommand{\func}[1]{\textbf{function }\texttt{#1}}
\newcommand{\call}[1]{\texttt{#1}}
\newcommand{\tabul}[1]{\hspace*{#1em}\hspace*{#1em}}
\newcommand{\mywhile}{\textbf{while}}
\newcommand{\mydo}{\textbf{do}}
\newcommand{\myuntil}{\textbf{until}}
\newcommand{\myswitch}{\textbf{switch}}
\newcommand{\myif}{\textbf{if}}
\newcommand{\myendif}{\textbf{endif}}
\newcommand{\mythen}{\textbf{then}}
\newcommand{\myelse}{\textbf{else}}
\newcommand{\myfor}{\textbf{for each}}
\newcommand{\myrepeat}{\textbf{repeat}}

\newcommand{\myendfor}{\textbf{endfor}}
\newcommand{\myendwhile}{\textbf{endwhile}}
\newcommand{\myor}{\textbf{or}}
\newcommand{\myand}{\textbf{and}}
\newcommand{\myno}{\textbf{no}}
\newcommand{\myret}{\textbf{return}}
\newcommand{\myendf}{\textbf{endfunction }}
\newcommand{\myendp}{\textbf{endprocedure }}



\def\odiv{{ \ominus\hspace{-8pt}:}\;}
\def\odivsmall{{ \ominus\hspace{-7.5pt}:}\;}
\def\smallodiv{{ \ominus\hspace{-7.45pt}:}\;}


%%%%%%%%%%%%%%%%%%%%%%%%%%%%%%%%%%%%%%%%%%%
%%%%%%%%%%%%%%%%%%%%%%%%%%%%%%%%%%%%%%%%%%%
\begin{document}
\begin{frontmatter}





\title{Residuation for Lexicographic Orders}


		\author[pisa]{Fabio Gadducci}
		\ead{fabio.gadducci@unipi.it}
		\author[perugia]{Francesco Santini\corref{cor1}}
		\ead{francesco.santini@unipg.it}
		
		\address[pisa]{Dipartimento di Informatica, Universit{\`a} degli Studi di Pisa, Italy}	
		\address[perugia]{Dipartimento di Matematica e Informatica, Universit{\`a} degli Studi di Perugia, Italy}	
		\cortext[cor1]{Corresponding author.}



\begin{abstract} 
	Residuation theory concerns the study of partially ordered algebraic structures, most often just monoids,
	equipped with a weak inverse for the monoidal operator.
	%
	One of its area of application is constraint programming, whose 
	key requirement is the presence of an aggregator operator for combining preferences.
	%
	Given a residuated monoid of preferences, this paper shows how to build a new residuated monoid 
	of (possibly infinite) tuples based on lexicographic order. 
\end{abstract}
        \begin{keyword}
        	Residuation theory, Lexicographic orders, Soft constraints. %\sep Bucket Elimination.
        \end{keyword}
    \end{frontmatter}





\section{Introduction}\label{sec:intro}
Residuation theory~\cite{residuation} concerns the study of partially ordered algebraic structures, most often just monoids, 
equipped with an operator that behaves as a weak inverse to the monoidal one, without the structure being necessarily a group.
%Such structures have since long be investigated in mathematics and computer science. %Concerning e.g. logics, residuated monoids form the basis for the semantics of substructural logics~\cite{onoETC}.
%As for e.g. discrete event systems such as weighted automata, the use of tropical semirings put forward the adoption of residuals for 
%the approximated solution of inequalities~\cite{resbook}. 
%
%Related to the latter, o
One of the recent area of application of residuation theory is constraint programming. 
%
Roughly, a soft constraint satisfaction problem
%(\emph{SCSP}) 
is given by a relation on a set of variables and a preference 
score to each assignment of the variables~\cite{jacm97,schiex}. 
%
A key requirement is the presence of an aggregator operator for combining preferences, making their set a monoid: a large body 
of work has been devoted to enrich such a structure, guaranteeing that resolution techniques are generalised by 
a parametric formalism for designing metrics and algorithms.
%
%An example are \emph{local-consistency} algorithms~\cite{arconsistency}, devised for safely moving costs  towards constraints involving a smaller number of variables, 
%without changing the set of solutions and their preference.  In order to ``move'' quantities, we need to  ``subtract'' costs somewhere and ``add'' them elsewhere. 

%The paper focuses on residuated monoids for constraint programming. 
The relevance of residuated monoids for local consistency algorithms, where in order to ``move'' costs they must be ``subtracted'' somewhere and ``added'' elsewhere,
 has been spotted early on~\cite{residuation1,resCS}, and various extensions  have been proposed~\cite{ipl}.
 Also, residuated monoids found applications in languages based on the Linda paradigm, such as 
soft concurrent constraint programming,  where a process  ``tells'' and ``asks'' constraints to a centralised store~\cite{labelled}. 
%We consider lexicographic orders, as used in contexts with multi-objective
%problems. That is, the preference values are obtained by the combination of separate concerns, and the order of the combination matters. 
%We identify suitable conditions to show how to effectively build 
%On the other side, we introduce a soft version of \emph{Bucket} and \emph{Mini-bucket} elimination algorithms, well-known exact and approximated techniques for inference,
%which exploits the presence of a residuated monoid: in order to have an estimation of  the approximation on the preference level of a solution, it is necessary to use a removal operator. Finally we present a \emph{Depth-First Branch-and-Bound} algorithm, which exploits upper and lower bounds to prune search. 
%Our proposals generalise the original soft versions of these approximation techniques presented in~\cite{bucketsemiring}.
%
%[OLD STUFF]
%Preference structures in the Constraint-programming literature have been historically equipped with a monotone aggregation-operator~\cite{chapterrossi}: when \emph{soft} constraints are composed in a \emph{Soft Constraint Satisfaction Problem} (\emph{SCSP}), the cost, or alternatively the reward, for satisfying such constraints needs to be aggregated.\footnote{We instead refer to \emph{hard} constraints, when constraints can be either ``fully'' satisfied, or just not satisfied, thus having no associated preference.} In soft constraints,
%an assignment of the variables on which the constraint relation is defined, is associated with a preference score. Some well-known examples of such preference structures in constraints are [2] and [22]. Besides these two seminal works, a large body of work has been devoted to the study of algebraic structures for sets of preferences, guaranteeing that techniques can be generalised by having a parametric and formal tool to design metrics and operators
%
%
%Some  well-known  examples of such preference structures in constraints  are \cite{jacm97}  and \cite{schiex}. Besides these two seminal works, a  large body of work has been devoted to the study of algebraic structures for sets of preferences, guaranteeing that  techniques can be generalised by having a parametric and formal tool to design metrics and operators.
%
%The motivation  is due to its use in algorithms that solve CSPs: for instance, the idea behind \emph{local-consistency} algorithms~\cite{arconsistency} is to safely move costs  towards smaller arity constraints, without changing the set of solutions and their preference.  Hence, we need to ``subtract'' costs somewhere, and then ``add'' them somewhere else.
%
%In \cite{ipl} we have extended \cite{residuation1} with the purpose to have  an inverse operator to work on bipolar preference values (i.e., positive and negative ones). In \cite{labelled} we have used residuation to give a labelled semantics to a process algebra \emph{telling} and \emph{asking} soft constraints to a centralised store. In  \cite{catuscia}, some of the authors of this paper provided a novel formalisation of soft constraints in terms of \emph{polyadic algebras} whose carrier is a residuated monoid.
%
%In this paper  we turn our attention to the property of residuation in lexicographic orders.  
Lexicographic orders are useful in applications that involve multiple objectives and attributes,
and they have been extensively investigated in the literature on soft constraints.
%,  where feasibility constraints impose restrictions on assignments that are actually possible.  
%\emph{Multi-objective combinatorial optimisation} (MOCO) is concerned with finding optimal solutions of systems as  $\min_{x \in  X} f(x) = (f_1(x), f_2(x), \dots, f_p(x))$, where $X$ is a feasible set of solutions, and $f_i$ are the objectives, $f_i: X \longrightarrow \mathbb R$ in a vector $f$.
%Lexicographic orders in SCSPs have been extensively investigated in the literature. 
However, usually the connection has been established by encoding a lexicographic \emph{hard} constraint problem, where the preference structure is a Boolean algebra, 
into a soft constraint formalism. 
%
%For example, in \cite{freuderlexi} the authors show how to encode a lexicographic order and how the resulting structure
%can support specialised algorithms such as \emph{Branch-and-bound}. 
%In \cite{jheuristics}  the expressivity distance between soft constraints embedded in 
%lexicographic orderings is measured against \emph{CP-nets}: the core notions exploited by the (conditional) CP-nets model are  \emph{ceteris paribus}  
%statements, e.g., ``I prefer red wine to white wine when meat is served''.
%
%In \emph{Constraint Logic Programming}~\cite{clp},  in place of \emph{unification}  constraints are accumulated and tested for satisfiability over the common domain where they are interpreted.  
%\emph{Hierarchical Constraint Logic Programming}~\cite{hierarchicalclp} frameworks allow to handle %over-constrained problems by considering 
%both hard constraints  and several preference levels of soft constraints, whose violations need to be minimised, and such levels are usually managed 
%following a lexicographic order~\cite{valuation}. 
%For example, in \cite{valuation} the authors  exploit the lexicographic ordering in constraint hierarchies by mapping layers to partially-ordered valuation structures. 
%
However, while lifting the algebraic structure of a preference set to the associated set of (possibly infinite) tuples with a point-wise order is straightforward, 
doing the same for the lexicographic order is not, and it cannot be directly achieved for the formalisms in \cite{jacm97,schiex}. 
%
The solution advanced in~\cite{GadducciHMW13,valuation} is to drop some preference values from the domain carrier of the set of tuples. 
%
The present work builds on
this proposal by dealing with sets of preferences that form residuated monoids, systematising and extending the case of infinite tuples tackled in~\cite{sca} 
to tuples of any length.


The paper extends those results in~\cite{jelia2021} concerning lexicographic orders, and it has the following structure. 
In Section~\ref{sec:om} we present  the background on partially ordered residuated monoids, the structure 
we adopt to model preferences. 
%
In Section~\ref{sec:collapsing} we consider the collapsing elements of a monoid, which are used to define an ad-hoc algebraic structure representing 
(possibly infinite) lexicographically ordered tuples of elements of the chosen monoid, as given in Section~\ref{sec:lexico}. 
The latter section also presents our main construction, introducing residuation for these lexicographically ordered monoids.  
Finally, in Section~\ref{sec:conclusion} we wrap up the paper with concluding remarks and ideas about future works.
%

\section{Preliminaries}\label{sec:om}

%In contrast to Boolean constraints, a soft constraint is a constraint that needs 
%not be fully satisfied~\cite{jacm97}.
%nstead, such a constraint assigns a \emph{preference value} that measures 
%the degree of satisfaction of a solution.

This section recalls some of the basic algebraic structures needed for defining
the set of preference values. We propose elements of \emph{ordered monoids} 
to serve as preferences, which allows us to compare and compose preference values.



\subsection{Ordered Monoids}
\label{sec:lem}

The first step is to define an algebraic structure for modelling preferences. 
We refer to~\cite{ipl} %for the missing proofs as well as 
for an introduction and a comparison with other proposals.
% such as~\cite{xxx}\todo{Manca citazione}.

\begin{definition}[Orders]
	A partial order (PO) is a pair $\langle A, \leq \rangle$ such that
	$A$ is a set %of values 
	and $\leq \,\,\subseteq A \times A$ is a reflexive, transitive, and
	anti-symmetric  relation.
	% and $\forall a \in A. \bot\leq a$.
	%
	%A partial order with bottom (POT) is a triple
	%$\langle A, \leq, \bot \rangle$ such that $\langle A, \leq \rangle$ is a PO and
	%$\forall a \in A. \bot \leq a$.
	%
	A join semi-lattice (simply semi-lattice, SL) is a POs
	such that any finite subset of $A$ has a least upper bound (LUB);
	a complete lattice (CL) is a PO such that any subset of A has a LUB.
\end{definition}

%We write 
The LUB of a subset $X \subseteq A$ is denoted $\bigvee X$, and it is unique. 
Note that we require the existence of $\bigvee \emptyset$, which is the bottom
of the order, denoted as $\bot$, and sometimes we will talk about a PO with bottom element
(POB).
The existence of LUBs for any subset of $A$ (thus including $\emptyset$) guarantees that 
CLs also have greatest lower bounds (GLBs) for any subset $X$ of $A$: 
it will be denoted by $\bigwedge X$. 
Whenever it exists, 
%and it always does in a CL, 
$\bigvee A$ corresponds to the top of the order, denoted as $\top$.
%

%We considered the LUBs of possibly infinite sets just for the sake of simplicity: 
%our proposal would fit also the finite case.
%
%Obviously, Ls also have the greatest lower bound for any subset $Y \subseteq A$.
%In the following we fix a BL ${\mathbb L} = \langle A, \leq, \1 \rangle$.

%\begin{definition}[compact elements]
%An element $a \in A$ is compact (or finite) if whenever $a \leq \bigvee Y$ there exists a finite subset
%$X \subseteq Y$ such that $a \leq \bigvee X$.
%%
%%Let $A^C \subseteq A$ be the set of compact elements of ${\mathbb C}$.
%%Then ${\mathbb C}$ is algebraic if $\forall c \in A. c = \bigvee \{ d \in A^C \mid d \leq c\}$.
%\end{definition}


%Note that for complete lattices the definition of compactness given above coincides with the one using
%directed subsets. It will be easier to generalize it, though, to compactness with respect to the monoidal operator (see Def.~\ref{def:compactness}).
%
%We let $A^C \subseteq A$ denote the set of compact elements of ${\mathbb C}$. Note however
%that $A^C$ might be trivial: indeed, in the the segment $[0, 1]$ of the reals
%with the usual order, only $0$ is a compact element. As we are going to see, the situation for the soft paradigm
%can be more nuanced.
%\marginpar{is algebraicity needed?}
%

\begin{definition}[Ordered monoids]\label{defn:clm}
	A (commutative) monoid is a triple
	$\langle A, \monop,$ $\1 \rangle$ such that $\monop: A \times A \rightarrow A$ is
	a commutative and associative function and $\1 \in A$ is its \emph{identity} element,
	i.e., $\forall a \in A. a \monop \1 = a$.
	
	A partially ordered monoid (POM) is a 4-tuple
	$\langle A, \leq, \monop, \1 \rangle$ such that 	
	$\langle A, \leq \rangle$ is a PO and $\langle A, \monop, \1 \rangle$ a monoid.
	%
	A semi-lattice monoid (SLM) and a complete lattice monoid (CLM) are 
	POMs such that their underlying PO is a SL, a CL respectively.
\end{definition}

For ease of notation, we use infix operators: $a \monop b$ stands for $\monop(a,b)$.

\begin{example}[Powerset]\label{ex:powerset}
	Given a (possibly infinite) set $V$ of variables, we consider
	the monoid $\langle 2^V, \cap, V \rangle$
	of (possibly empty) subsets of $V$, with intersection as the monoidal operator.
%	Since the operator is idempotent (i.e., $\forall a\in A.\, a \monop a = a$), 
%	the natural order ($\forall a, b \in A.\, a \leq b$ iff $a \monop b = b$) 
%	is a partial order, and 
%	it coincides with subset inclusion:
	The partial order is given by subset inclusion, and 
	$\mathcal{P} = \langle 2^V, \subseteq, \cap, V \rangle$
	is a CLM, with $\emptyset$ and $V$ the bottom and top elements. % yet not necessarily a CLM.
\end{example}

\subsection{Distributive monoids}

In general, the partial order $\leq$ and the monoid operator $\otimes$ can be unrelated.
This is not the case for distributive CLMs.

\begin{definition}[Distributivity]
	\label{dist}
	A SLM $\langle A, \leq, \monop, \1 \rangle$ is finitely distributive if
	$\forall X \subseteq A.\, \forall a \in A.\,  a \monop  \bigvee X = \bigvee \{a \monop x \mid x \in X\}$ 
	whenever $X$ is finite.
	A CLM is distributive if the equality holds also whenever X is infinite. % and the LUBs exist.
\end{definition}

%Being distributive means that the equality hold whenever both side are defined.
%As for CLMs, the appropriate LUBs always exist.
In the following, we will sometimes write $a \otimes X$ for the set  $\{a \monop x \mid x \in X\}$.

\begin{remark}
	Note that $a \leq b$ is equivalent to $\bigvee \{a,b\} = b$ for all $a, b \in A$.
	Hence, finite distributivity implies that $\otimes$ is monotone with respect to $\leq$ 
	(i.e., $\forall a, b, c \in A.\, a \leq b \Rightarrow a \otimes c \leq  b \otimes c$)
	and that $\bot$ is the zero element
	of the monoid (i.e., $\forall a \in A.\, a\otimes  \bot = \bot$).
	The powerset CLM $\mathcal{P}$ in Example~\ref{ex:powerset} is distributive.
\end{remark}

\begin{example}[Extended integers]\label{ex:bipolar}
	The distributive CLM of extended integers $\mathcal{Z} = \langle \mathbb{Z} \cup \{\pm \infty\}, \leq, +, 0 \rangle$
	enriches the usual order and addition operator requiring that for all
	$k \in \mathbb{Z}$ we have $-\infty < k < +\infty$ and $\pm\infty + k = \pm \infty$
	and furthermore that $\pm\infty + (-\infty) = -\infty$ and $+\infty + +\infty = +\infty$, so that
	$+\infty$ and $-\infty$ are respectively the top and the bottom element of the CL.
\end{example}

%We now state a fundamental property of the least upper bound:
% 
% \begin{lemma}
% 	\label{lem:join_ineq}
% 	Let $X,Y \subseteq A$ be subsets of a CLM $\langle A, \leq \mult, \1 \rangle$.
% 	If for every $x \in X$, there exists some $y \in Y$, such that $x \leq y$, then $\bigvee X \leq \bigvee Y$.
% \end{lemma}
% 
% \begin{proof}
% 	The premise implies that that every upper bound of $Y$ is an upper bound of $X$, which shows that $\bigvee X \leq \bigvee Y$.
% \end{proof}

\begin{remark}\label{remTrop}
	Finitely distributive SLMs precisely corresponds to \emph{tropical} semirings 
	by defining the (idempotent) sum operator as
	$a \oplus b = \bigvee \{a, b\}$ for all $a,b \in A$.
	Moreover, if $\1$ is the top of the SLM we end up 
	with \emph{absorptive} semirings~\cite{golan}, 
	which are known as $c$-semirings 
	in the soft constraint jargon~\cite{jacm97}.
	Together with monotonicity, imposing $\1$ to coincide with $\top$ means 
	that preferences are negative (i.e., $a \leq \1$ for all $a \in A$).
	Distributive CLMs are also known as \emph{quantales}~\cite{quantales}.
\end{remark}

\begin{remark}
	\label{remark}
	Given two (distributive) CLMs, it is easy to show that their Cartesian product, whose elements are pairs and where the partial order and the monoidal operator are 
	defined point-wise, is a (distributive) CLM.
	%
	In particular, in the following we consider the Cartesian product $\mathcal{Z}^2 = \mathcal{Z} \times \mathcal{Z}$:
	%of $\langle \mathbb{Z} \cup \{\pm \infty\}, \leq, +, 0 \rangle$ with itself: 
	its set of
	elements is $(\mathbb{Z} \cup \{\pm \infty\})^2$, the identity element is $(0,0)$,
	and the top and bottom elements are $(+\infty, +\infty)$ and $(-\infty, -\infty)$, respectively.
\end{remark}

\subsection{Residuated monoids}

We introduce \emph{residuation}, which allows us to define a ``weak'' inverse 
operator with respect to the monoidal operator $\otimes$. 
In this way, besides aggregating values together, it is also possible to remove one from 
another. Residuation theory~\cite{golan} is concerned with the study of sub-solutions of the 
equation $b \otimes x = a$, where $x$ is a ``divisor'' of $a$ with respect to $b$. 
The set of sub-solutions of an equation contains also the 
possible solutions, whenever they exist, and in that case the maximal element is also a 
solution.

\begin{definition}[residuation]
	A residuated POM is a 5-tuple $\langle A,$ $\leq, \otimes,  \odiv, \1 \rangle$ such that
	$\langle A, \leq \rangle$ is a PO,  $\langle A, \otimes, \1 \rangle$ is a monoid, and
	$\odiv: A \times A \rightarrow A$ is a function such that $\forall a, b, c \in A.\ b \otimes c \leq a \iff c \leq a \odiv b$.
\end{definition}

In the following, we will sometimes write $a \odiv X$ and $X \odiv a$
for the set  $\{a \odiv x \mid x \in X\}$ and  $\{x \odiv a \mid x \in X\}$,
respectively.

\begin{remark}\label{natural}
	It is easy to show that residuation is monotone on the first argument and
	anti-monotone on the second. In fact, in an SLM 
	$\bigvee (X \odiv a) \leq \bigvee X \odiv a$,
	and the same in a CLM also with respect to infinite subsets.
	However, the equality does not hold,
	%as suggested by Lemma~\ref{someProps},
	e.g. in the Cartesian product $\mathcal{N}^2$ of the CLM  
	$\mathcal{N} = \langle \mathbb{N} \cup \{\infty\}, \geq,$ $+, 0 \rangle$ with itself.
	Also, 
	$a \odiv \bigvee X \leq \bigwedge (a \odiv X)$
	whenever the latter exists. % as it does in CLMs.
\end{remark}

\begin{example}
Let us consider the powerset SLM $\mathcal{P}$ of Example~\ref{ex:powerset}. 
Given subsets $A$ and $B$ of $V$, their residuation $A \odiv B$ 
is the largest $C$ such that $B \cap C \subseteq A$, that is $(V \setminus B) \cup A$. From the point of view
of the set-theoretical interpretation of boolean logics, it corresponds to the semantics of the formula $B \rightarrow A$.
\end{example}

\begin{remark}
	\label{remarkC}
	As it holds for distributive POMs, given two residuated POMs their Cartesian product
	is a residuated POM, with $\odiv_{\mathcal{A} \times \mathcal{B}}$ defined point-wise.
\end{remark}

We can give an order characterisation of the residuation operator.

\begin{lemma}\label{rclm1}
	Let $\langle A,$ $\leq, \otimes,  \odiv, \1 \rangle$ be a residuated POM with bottom.
	Then $a \odiv b = \bigvee \{c \mid b \otimes c \leq a\}$ for all $a, b \in A$.
\end{lemma}

In words, the LUB of the (possibly infinite) set 
$\{c \mid b \otimes c \leq a\}$ exists and is equal to $a \odiv b$.
%
In fact, residuation implies distributivity (see~\cite[Lemma 2.2]{ipl}).

\begin{lemma}\label{rclm2}
	Let $\langle A, \leq, \monop, \odiv, \1 \rangle$ be a residuated POM. 
	Then $\monop$ is monotone.
	If additionally it is a SLM (CLM), then it is finitely distributive
	(distributive).
\end{lemma}
%\begin{proof}
%xxx
%\end{proof}

In view of Lemma~\ref{rclm1} and Lemma~\ref{rclm2}, it is immediate to
establish a one-to-one correspondence between residuated CLMs and distributive CLMs.

%Conversely, it is noteworthy that CLMs are always residuated,
%and the following folklore fact holds.
%%
%%The proofs of the following lemmas come from \cite{catuscia,ipl}.
%
%\begin{lemma}
%	\label{resCLM}
%	Let $\langle A, \leq, \monop, \1 \rangle$ be a distributive CLM. 
%	%
%	Then it is residuated and 
%	$\forall a, b \in A.\ a \odiv b = \bigvee \{c \mid b \otimes c \leq a\}$.
%\end{lemma}

\smallskip
We close with a simple lemma relating residuation with the top and the bottom elements of a POM.

\begin{lemma}
	\label{someProps}
	Let $\langle A, \leq, \monop, \odiv, \1 \rangle$ be a residuated POM. 
	If it has the bottom element $\bot$, then it also has the top element $\top$and
	$\forall a \in A.\ a \odiv \bot = \top$.
	%
	Viceversa, if it has the top element $\top$ then
	$\forall b \in A.\ \top \odiv b = \top$.
\end{lemma}

\begin{remark}
\label{usualNat}
	Nothing can be stated for $\bot \odiv a$, since there could be elements that are 
	$\bot$-divisors: see the CLM $\mathcal{N}^2$ 
	in Remark~\ref{natural}, where 
	$\langle \infty, 3\rangle  \otimes \langle 4, \infty \rangle = \langle \infty,\infty \rangle$. 
	
	Similarly, nothing can be stated for $a \odiv \top$: see the Cartesian 
	product of the CLM $\mathcal{N}$ with its dual CLM
	$\mathcal{O} = \langle \mathbb N \cup \{\infty\}, \leq, +, 0 \rangle$.
\end{remark}

%(see e.g.~\cite{ecai06} 
%for a brief survey on residuation for such semirings)
%\marginpar{sistemare da qualche parte}
%\begin{definition}
%Let $C$ be a sub-monoid of $A$. $C$ is an ideal of $A$ if 
%$\forall a \in A, c \in C.\ a \otimes c \in C$.
%\end{definition}

\section{The ideal of collapsing elements}
\label{sec:collapsing}
As shown in~\cite{GadducciHMW13}, the first step for obtaining SLMs based on a lexicographic order 
is to restrict the carrier of the monoid. 

\begin{definition}
	Let $\langle A, \monop, \1 \rangle$ be a monoid. Its sub-set $I(A)$ 
	of \emph{cancellative} elements is defined as 
	$\{ c \mid \forall a, b \in A.\ a \otimes c = b \otimes c \implies a = b \}$.
\end{definition}

We  recall a well-known fact.
% ($C(A)$ is the set of collapsing elements of $A$).

\begin{lemma}
	\label{ideal}
	Let $\langle A, \monop, \1 \rangle$ be a monoid.
	Then $C(A) = A\setminus I(A)$ is
	a prime ideal (and thus, $\langle C(A), \monop, \1 \rangle$
	and $\langle I(A), \monop, \1 \rangle$ are sub-monoids).
\end{lemma}

Explicitly, $C(A) = \{ c \mid \exists a, b \in A.\ a \neq b \wedge a \otimes c = b \otimes c\}$.
%
Being an ideal means that $\forall a \in A, c \in C( A).\ a \otimes c \in C(A)$,
and being prime further states that 
$\forall a, b \in A.\ a\otimes b \in C( A) \implies a \in C(A) \vee b \in C(A)$.
%
All the proofs are straightforward, and we denote $C(A)$ as the set of \emph{collapsing} 
elements of $A$. 

Note that an analogous closure property does not hold for LUBs. With an abuse of notation,
for a POM $\mathcal{A}$ we will often denote as $C(\mathcal{A})$ and $I(\mathcal{A})$ the 
sets of collapsing and cancellative elements, respectively.

\begin{example}\label{flat}
	Consider
	%\todo{S.: Non ho trovato. p. 4, Ex. 4, $C(\mathbb{N} \times \mathbb{N})$: Is $C((\mathbb{N} \cup \{ \infty \})^2)$ meant?  
	%Better give the monoid a name, e.g., $N$ and use this. G.: \`E diventato esempio 5.}  
	the monoid of natural numbers $\langle \mathbb N, +, 0 \rangle$ and 
	the (non distributive) CLM $\mathcal{M} = \langle \mathbb{N} \cup \{\bot,\top\}, \emptyset, +, 0\rangle$ 
	obtained by lifting the flat order (i.e., $a \not \leq b$ for any $a, b \in \mathbb N$ and 
	$a + \bot = \bot = \bot + \bot$ and $a + \top = \top$  for any $a \in \mathbb N \cup \{\top\}$).
	We have $I(\mathcal{M}) =  \mathbb{N}$, yet $\langle \mathbb N, \emptyset, +, 0\rangle$ is not closed under LUBs. 
	
	Now, let us consider the distributive CLM
	$\mathcal{O} = \langle \mathbb N \cup \{\infty\}, \leq, +, 0\rangle$ 
	obtained by lifting the usual order (see Remark~\ref{usualNat}).
	We have $I(\mathcal{O}) = \mathbb{N}$ and $\langle \mathbb N, \leq, +, 0\rangle$ is a finitely distributive SLM, yet
	it is not closed under infinite LUBs.
\end{example}

We now present a simple fact on Cartesian products  that is needed later on.

\begin{lemma}
	Let $\mathcal{A}$, $\mathcal{B}$ be POMs.
	Then $C(A \times B) = (C(A) \times B) \cup (A \times C(B))$. 
\end{lemma}

\begin{example}
	Let us consider again the CLM $\mathcal{N} = \langle \mathbb N \cup \{\infty\}, \geq, +, 0 \rangle$ 
	of Remark~\ref{natural} and the Cartesian product $\mathcal{N}^2$. 
	Now $C(\mathcal{N}^2)$ is not closed under finite LUBs: it suffices to consider 
	$X= \{\langle \infty, 3 \rangle, \langle 4, \infty \rangle\} \subseteq C(\mathcal{N}^2)$,
	since $\bigvee X  = \langle 3,4 \rangle \not \in C(\mathcal{N}^2)$.
	%
	Neither is $C(\mathcal{N}^2)$ closed under residuation, 
	%as suggested by 
	%Lemma~\ref{someProps}, 
	since the top element is not collapsing.
	Indeed, in $C(\mathcal{N}^2)$ we have 
	$\langle \infty, 4 \rangle  \odiv \langle \infty, 3 \rangle = \langle 0,1\rangle$.
\end{example}

\begin{remark}
	In an absorptive CLM $\mathcal{A}$ we have that $a \odiv b = 1$ 
	whenever $b \leq a$. Hence $C(\mathcal{A})$ is usually not closed under residuation, since 
	$1$ is cancellative.
\end{remark}

\subsection{A different view on collapsing elements}

When the first presentation of lexicographic 
SLMs was provided~\cite{GadducciHMW13}, a different 
set of collapsing elements was considered.

\begin{definition}[\cite{GadducciHMW13}]
	Let $\langle A, \leq, \monop, \1 \rangle$ be a POM. Its sub-set $C'(A)$ 
	is defined as 
	$\{ c \mid \exists a, b \in A.\ a < b \wedge a \otimes c = b \otimes c\}$.
\end{definition}

Clearly, $C'(A) \subseteq C(A)$. However, we can replicate Lemma~\ref{ideal}.
%
%First, we say that $\times$ is tone preserving i 

\begin{lemma}\label{lemma:six}
	Let $\langle A, \monop, \1 \rangle$ be a monoid
	and $I'(A) = A \setminus C'(A)$.
	%
	Then $C'(\mathcal{A})$ is an ideal 
	(and thus, $\langle C'(A), \monop, \1 \rangle$ is a sub-monoid).
	If $\monop$ is monotone, then 
	$C'(\mathcal{A})$ is a prime ideal
	(and thus, also $\langle I'(A), \monop, \1 \rangle$ is a sub-monoid).
\end{lemma}

Explicitly, $I'(A) = \{ c \mid \forall a, b \in A.\ a \otimes c = b \otimes c \implies a \not < b \}$.
%
The definitions we encounter in the next section 
could then be rephrased using 
$I'(A)$ and $C'(A)$ with minimal adjustments, thus confirming the proposal
in~\cite{GadducciHMW13}.\footnote{And in fact, the Lemma~\ref{lemma:six} holds also for 
	a property that is weaker than monotonicity: it suffices that 
	$\forall a, b, c.\ a \leq b \implies (a \otimes c \leq b \otimes c) \vee (b \otimes c \leq a \otimes c)$.
}

However, what is in fact noteworthy is that the two approaches are coincident
whenever distributivity holds, as shown by  Lemma~\ref{lemma:seven}.
%\todo{dire dove differiscono}

\begin{lemma}\label{lemma:seven}
	Let $\langle A, \leq, \monop, \1 \rangle$ be a finitely distributive SLM.
	%
	Then $C'(A) = C(A)$.
\end{lemma}
\begin{proof}
	We already noted that $C'(A) \subseteq C(A)$ always holds. Now, let $a, b, c$ such that
	$a \neq b \wedge a \otimes c = b \otimes c$: it suffices to consider $a \vee b$, noting 
	that it must be either $a < a \vee b$ or $b < a \vee b$ and that by distributivity 
	$(a \vee b) \otimes c = a \otimes c = b \otimes c$.
\end{proof}

\begin{remark}
	Consider the (non distributive) CLM $\mathcal{S} = \langle \mathbb [0 \ldots n] \cup \{\bot,\top\}, \emptyset, +, 0 \rangle$ 
	obtained by lifting the initial segment $[0 \ldots n]$ of the natural numbers with the flat order (as done for
	the CLM of natural numbers in Example~\ref{flat}). Here addition is capped, so that 
	e.g. $n + m = n$ for all $m$. Hence, $C(\mathcal{S}) = [1 \ldots n] \cup \{\bot,\top\}$
	that is, all elements except $0$. Instead, $C'(\mathcal{S}) =  \{\bot,\top\}$.
\end{remark}

\section{On lexicographic orders}\label{sec:lexico}
We now consider lexicographic orders, exploiting the results in Section~\ref{sec:collapsing}.
%
We first discuss such orders for binary products, then we move to iterated products, 
and we close with $\omega$-iterated products, i.e., tuples of infinite length.

\subsection{Binary products}\label{sec:bynary}

This section reframes and streamlines For the new definition of collapsing elements some results originally
introduced in~\cite{GadducciHMW13}. Given a partial order $\leq$ we denote by $<$ the associated strict 
partial order: the former can be recovered from the latter by closing with respect to reflexivity.

\begin{proposition}\label{prop:lexBI}
	Let $\mathcal{A}$, $\mathcal{B}$ be POMs with bottom elements $\bot_\mathcal{A}$ and $\bot_\mathcal{B}$.
	%
	Then the tuple $Lex(\mathcal{A}, \mathcal{B}) = 
	\langle A \times_L B, \leq_L, \monop_{\mathcal{A} \times \mathcal{B}}, \1_{\mathcal{A} \times \mathcal{B}} \rangle$
	is a POM with bottom element $\bot_{\mathcal{A} \times \mathcal{B}}$,
	where $A \times_L B = I(A) \times B \cup C(A) \times \{\bot_\mathcal{B}\}$ and
%	\begin{itemize}
		%\item 
		$\langle a_1, b_1\rangle <_L \langle a_2, b_2\rangle$ if $a_1 <_\mathcal{A} a_2$ or $a_1 = b_1$ 
		and  $a_2 <_\mathcal{B} b_2$.
%	\end{itemize}
\end{proposition}

\begin{proof}
	Monoidality of $\monop_{\mathcal{A} \times \mathcal{B}}$ is given by being $I(\mathcal{A})$ a sub-monoid
	and $C(\mathcal{A})$ an ideal of $\mathcal{A}$, while $\leq_L$ being a partial order is straightforward.
\end{proof}

Note that $Lex(\mathcal{A},\mathcal{B})$ is a sub-monoid of the Cartesian product $\mathcal{A} \times \mathcal{B}$,
hence with the same monoidal operator $\monop_{\mathcal{A} \times \mathcal{B}}$ as well as 
bottom $\bot_{\mathcal{A} \times \mathcal{B}} = \langle \bot_\mathcal{A}, \bot_\mathcal{B} \rangle$ and
identity $\1_{\mathcal{A} \times \mathcal{B}} = \langle \1_\mathcal{A}, \1_\mathcal{B} \rangle$,
while instead $\leq_{\mathcal{A} \times \mathcal{B}}$ is contained in $\leq_L$.
%and the definitions of $\monop$, $\1$, and $\bot$ coincide.
%
%Also, the bottom element is needed for padding the tuples, in order to make
%simpler the definition of the order.

\begin{lemma}
Let $\mathcal{A}$, $\mathcal{B}$ be POMs with bottom elements $\bot_\mathcal{A}$ and $\bot_\mathcal{B}$, 
respectively.
Then $C(A \times_L B) = I(A) \times C(B) \cup C(A) \times \{\bot_\mathcal{B}\}$.
\end{lemma}

We now consider distributivity for binary lexicographic products.

\begin{proposition}\label{def:lexBI}
Let $\mathcal{A}$, $\mathcal{B}$ be finitely distributive SLMs (distributive CLMs).
Then so is $Lex(\mathcal{A}, \mathcal{B})$ and for any finite subset (any subset) $X \subseteq A \times_L B$ 
we have $\bigvee X = \langle \bigvee X_\mathcal{A}, \bigvee X_\mathcal{B} \rangle$ 
where $X_\mathcal{A} = \{ a \mid \langle a, b \rangle \in X\}$
and $X_\mathcal{B} = \{ b \mid \langle \bigvee X_\mathcal{A}, b \rangle \in X\}$.
\end{proposition}
\begin{proof}
	Now, $\bigvee X$ is clearly the LUB of $X$ with respect to $\leq_{\mathcal{A} \times \mathcal{B}}$,
	just that that $\bigvee X =  \langle \bigvee X_\mathcal{A}, \bot_\mathcal{B} \rangle$ if 
	$X_\mathcal{B} = \emptyset$, which may occur if $x <_\mathcal{A} \bigvee X_\mathcal{A}$ for all 
	$x \in X_\mathcal{A}$.

	Let us then move to distributivity. We need to show that
	$\bigvee \langle a, b\rangle \otimes_{\mathcal{A} \times \mathcal{B}} X = \langle a, b\rangle \otimes_{\mathcal{A} \times \mathcal{B}}  (\bigvee X)$ holds,
	with the latter equivalent to $\langle a \otimes_\mathcal{A} \bigvee X_\mathcal{A}, b \otimes_\mathcal{B} \bigvee X_\mathcal{B} \rangle$.
	Now, $\langle a, b\rangle \otimes_{\mathcal{A} \times \mathcal{B}} X$ coincides with 
	$Y = \{\langle a \otimes_\mathcal{A}  c,  b \otimes_\mathcal{B}  d \rangle \mid \langle c, d \rangle \in X\}$.
	By distributivity of $\mathcal{A}$, we have that $\bigvee Y_\mathcal{A} = a \otimes_\mathcal{A} \bigvee X_\mathcal{A}$.
	If $a$ is collapsing then we are done, since $b = \bot_\mathcal{B}$ and thus $b \otimes_\mathcal{B} \bigvee X_\mathcal{B} = 
	\bigvee Y_\mathcal{B} = \bot_{B}$
	even if either $X_\mathcal{B} \neq \emptyset$ or $Y_\mathcal{B} \neq \emptyset$. Otherwise, 
	consider $X_\mathcal{B}$ and note that 
	$$Y_\mathcal{B} = \{ c \mid \langle \bigvee Y_\mathcal{A}, c \rangle \in Y\}
	= \{ c \mid \langle a \otimes_\mathcal{A} \bigvee X_\mathcal{A}, c \rangle \in \langle a, b\rangle \otimes_{\mathcal{A} \times \mathcal{B}} X \rangle \}$$
	\noindent
	Now, note that $\bigvee X_\mathcal{A} \not \in X_\mathcal{A}$ implies $X_\mathcal{B} = \emptyset$ as well as, since $a$ is cancellative, $Y_\mathcal{B} = \emptyset$.
	If $X_\mathcal{B} = Y_\mathcal{B} = \emptyset$ we are done, so let us assume that $\bigvee X_\mathcal{A} \in X_\mathcal{A}$. Then
%	$$\{ b \otimes_\mathcal{B} z \mid z \in X'_\mathcal{B} \wedge 
%	      \langle a \otimes_\mathcal{A} \bigvee X_\mathcal{A}, b  \otimes_\mathcal{B} z \rangle \in \langle a, b\rangle \otimes_{\mathcal{A} \times \mathcal{B}} X \rangle \} =$$
	$$Y_\mathcal{B} = \{ b \otimes_\mathcal{B}   d \mid \bigvee X_\mathcal{A} \in X_\mathcal{A} \wedge d \in X_\mathcal{B} \wedge 
	      \langle a \otimes_\mathcal{A} \bigvee X_\mathcal{A}, b  \otimes_\mathcal{B} d \rangle \in \langle a, b\rangle \otimes_{\mathcal{A} \times \mathcal{B}} X \rangle \} =$$
	$$\{ b \otimes_\mathcal{B} d \mid  \bigvee X_\mathcal{A} \in X_\mathcal{A}  \wedge d \in X_\mathcal{B}\} =
	    b \otimes_\mathcal{B} \{ d \mid \langle \bigvee X_\mathcal{A}, d \rangle \in X \} = b \otimes_\mathcal{B} X_\mathcal{B}$$
	\noindent
	and thus, by the distributivity of $\otimes_\mathcal{B}$ we have $\bigvee Y_\mathcal{B} = b \otimes_\mathcal{B} \bigvee X_\mathcal{B}$.
\end{proof}

Since $Lex(\mathcal{A},\mathcal{B})$ is a distributive CLM if so are $\mathcal{A}$ and $\mathcal{B}$, it is also residuated.

\begin{example}
	Let us consider the tropical CLM $\mathcal{N}$ of natural numbers with inverse order 
	(see Remark~\ref{natural}),
	and the CLM $Lex(\mathcal{N},\mathcal{N})$. Clearly $C(\mathcal{N}) = \{+\infty\}$. We then have for example
	that
	$(3, 6) \odiv_L (4, 2) = \bigvee \{(x, y) \mid (4 + x, 2 + y) \leq_L (3, 6)\} = (0,0)$.
	Indeed, $(4 + x, 2 + y) \leq_L (3, 6)$ holds for any possible choice of $(x, y)$,
	since $4 + x <_\mathcal{N} 3$ for all $x$, hence $(0,0)$ as the result.
	%
	Note that we would have obtained $(0, 4)$ for the CLM $\mathcal{N}^2$ of the Cartesian product.
\end{example}

Indeed, this can be proved in general for POMs. As a shorthand, given a residuated POM 
and an element $a$, we denote $D(a) = \{ b \mid b \otimes (a \odiv b) = a\}$.

\begin{proposition}\label{def:resBI}
Let $\mathcal{A}$, $\mathcal{B}$ be residuated POMs with bottom elements $\bot_\mathcal{A}$ and $\bot_\mathcal{B}$.
%
Then so is $Lex(\mathcal{A}, \mathcal{B})$ and for any pair $\langle a_1, b_1\rangle, \langle a_2, b_2\rangle \in A \times_L B$
we have 
%\small
\[
	\langle a_1, b_1\rangle \odiv_L \langle a_2, b_2\rangle = 	\begin{cases}
	\begin{array}{lcl}
	\langle a_1 \odiv_\mathcal{A} \, a_2, b_1  \odiv_\mathcal{B} \, b_2\rangle 
	& & \mbox{if  }  a_1 \odiv a_2 \in I(\mathcal{A}) \mbox{ and } a_2 \in D(a_1)\\
	\langle a_1 \odiv_\mathcal{A} \, a_2, \top_\mathcal{B} \rangle 
	& & \mbox{if  } a_1 \odiv a_2 \in I(\mathcal{A}) \mbox{ and } a_2 \not \in D(a_1) \\
	\langle a_1 \odiv_\mathcal{A} \, a_2, \bot_\mathcal{B} \rangle 
	& & \mbox{if  }  a_1 \odiv a_2 \in C(\mathcal{A}) 
	\end{array}
	\end{cases}
\]
\end{proposition}
\begin{proof}
We need to prove that for all $\langle a, b\rangle \in A\times B$ we have
$$\langle a_2, b_2\rangle \otimes_{\mathcal{A} \times \mathcal{B}} \langle a, b\rangle \leq_L \langle a_1, b_1\rangle
\iff \langle a, b\rangle \leq_L  \langle a_1, b_1\rangle \odiv_L \langle a_2, b_2\rangle$$
%
Recall that $\otimes_{\mathcal{A} \times \mathcal{B}}$ works component-wise and 
$a_1 \odiv_\mathcal{A} a_2 = \bigvee \{a \mid a_2 \otimes_\mathcal{A} a \leq_\mathcal{A} a_1\}$.

\begin{itemize}
\item[$\Rightarrow$] Clearly $a \leq_\mathcal{A} a_1 \odiv_\mathcal{A} a_2$ holds.
If $a <_\mathcal{A} a_1 \odiv_\mathcal{A} a_2$ we are done, so let 
$a = a_1 \odiv_\mathcal{A} a_2$. If $a \in C(\mathcal{A)}$ then 
$b = \bot_\mathcal{B}$ and we are done, so let $a = a_1 \odiv_\mathcal{A} a_2 \in I(\mathcal{A)}$.
If $a_2 \not \in D(a_1)$ then by having chosen $\top_\mathcal{B}$ we are done, so let $a_2 \in D(a_1)$.
Then $a_2 \otimes_\mathcal{A} a = a_1$: it implies $b_2 \otimes b \leq_\mathcal{B} b_1$
and we are done.

\item[$\Leftarrow$]
Clearly $a_2 \otimes_\mathcal{A} a \leq_\mathcal{A} a_1$ holds.
If $a_2 \otimes_\mathcal{A} a <_\mathcal{A} a_1$ we are done, so let 
$a_2 \otimes_\mathcal{A} a = a_1$: it implies $a_2 \in D(a_1)$.
%If $a \in C(\mathcal{A)}$ then 
%$b = \bot_\mathcal{B}$ and we are done, so let $a \in I(\mathcal{A)}$.
If $a_1 \odiv_\mathcal{A} \, a_2 \in C(A)$ then $b = \bot_\mathcal{B}$ and we are done, so let
$a_1 \odiv_\mathcal{A} a_2 \in I(A)$. Then $b\leq_\mathcal{B} b_1 \odiv_\mathcal{B} \, b_2$
and we are done.
\end{itemize}
\end{proof}

While $\langle \bot_\mathcal{A}, \bot_\mathcal{B} \rangle$ is the bottom element, note that the top element can be 
either $\langle \top_\mathcal{A}, \bot_\mathcal{B} \rangle$ or $\langle \top_\mathcal{A}, \top_\mathcal{B} \rangle$:
the former if $\top_\mathcal{A} \in C(\mathcal{A})$, the latter if $\top_\mathcal{A} \in I(\mathcal{A})$.

\subsection{Iterated orders}

We now move to finitely iterated products. For a POM 
$\mathcal{A} = \langle A, \leq_\mathcal{A}, \monop_\mathcal{A}, \1_\mathcal{A} \rangle$, 
we denote by $\mathcal{A}^k =  \langle A^k, \leq^k_\mathcal{A}, \monop^k_\mathcal{A}, \1^k_\mathcal{A} \rangle$
the POM of its Cartesian $k$-ary product, whose elements are tuples of length $k$
of elements in $\mathcal{A}$ and the monoidal operator is component-wise.
%
If $\mathcal{A}$ is residuated, so is $\mathcal{A}^k$ with $\odiv^k_\mathcal{A}$ component-wise.

\begin{proposition}\label{def:lexilist}
	Let $\mathcal{A}$ be a POM with bottom element $\bot_\mathcal{A}$
	and the family 
	$Lex^1(\mathcal{A}) = \mathcal{A}$ and $Lex^{k+1}(\mathcal{A}) = Lex(\mathcal{A}, Lex^k(\mathcal{A}))$.
	Then $Lex^k(\mathcal{A})$ is a POM with bottom element $\bot^k_\mathcal{A}$ for all $k$.
\end{proposition}

The definition of the family exploits binary products: the proof is immediate,
and it allows to lift distributivity and residuation from $\mathcal{A}$.
However, in order to later consider infinite tuples, we look for 
a more operational definition. % of such POMs.

\begin{proposition}\label{pomLex}
	Let $\mathcal{A}$ be a POM with bottom element $\bot_\mathcal{A}$.
	Then $Lex^{k}(\mathcal{A}) = \langle A^k_L, \leq^k_L, \monop^k_\mathcal{A}, \1^k_\mathcal{A} \rangle$
	%for all $k$
	where $A^k_L = \bigcup_{i < k}I(A)^i \times A \times \{\bot_\mathcal{A}\}^{k-1-i}$
	and $\langle a_1, \ldots, a_k\rangle <^k_L \langle b_1, \ldots, b_k\rangle$ if there exists $j \leq k$ such that
	$a_i = b_i$ for all $i < j$ and %either $j = k$ and $a_k \leq_\mathcal{A}  b_k$ or 
	$a_{j} <_\mathcal{A}  b_{j}$.
\end{proposition}
\begin{proof}
	The proof is by induction on $k$. Concerning the order, it is immediate.
	Thus, let us look at the set of elements. For $k = 1$ it is obvious.
	Let us assume it true for $k = n$, that is,
	$A^n_L = \bigcup_{i < n}I(A)^i \times A \times \{\bot_\mathcal{A}\}^{n-1-i}$ .
	Now, 
	$A^{n+1}_L =  I(A) \times A^{n}_L  \cup C(A) \times \{\bot_\mathcal{A}\}^{n} =
	\bigcup_{i < n}I(A)^{i+1} \times A \times \{\bot_\mathcal{A}\}^{n-1-i} \cup C(A) \times \{\bot_\mathcal{A}\}^{n} = %\newline
	\bigcup_{i < n+1}I(A)^{i} \times A \times \{\bot_\mathcal{A}\}^{n-i}$
	and we are done, where the latter equality holds since
	$I(A) \times \{\bot_\mathcal{A}\}^{n+1} \subseteq \bigcup_{i < n}I(A)^{i+1} \times A \times \{\bot_\mathcal{A}\}^{n-i}$
	and $A = I(A) \cup C(A)$.
\end{proof}

As for the binary product, the top element can be 
either $\top_\mathcal{A} \cdot \bot^{k-1}_\mathcal{A}$ or $\top^k_\mathcal{A}$,
depending if $\top_\mathcal{A} \in C(\mathcal{A})$ or $\top_\mathcal{A} \in I(\mathcal{A})$.
%
We can exploit the result above also to provide a simple characterisation of collapsing elements.

\begin{lemma}
	Let $\mathcal{A}$ be a POM with bottom element $\bot_\mathcal{A}$.
	Then $C(Lex^{k}(\mathcal{A})) = \bigcup_{i < k}I(A)^i \times C(A) \times \{\bot_\mathcal{A}\}^{k-1-i}$.
	%for all $k$.
\end{lemma}

%Note also that $\langle a_1, \ldots, a_k\rangle <^k_L \langle b_1, \ldots, b_k\rangle$ if there exists $j \leq k$ such that
%$a_i = b_i$ for all $i < j$ and $a_{j} <_\mathcal{A} b_{j}$.

We now move to distributivity. 
Given a tuple $t = \langle a_1, \ldots, a_k \rangle$ of elements in $A^k$, we denote with $t_i$ its 
$i$-th component $a_i$ and with $t_{\mid i}$ its prefix $\langle a_1, \ldots, a_i \rangle$,
with the obvious generalisation of the latter for a set $X \subseteq A^k$.
%noting that $a_1 = a_{\mid 1}$.

\begin{theorem}\label{theo:lexiSLM}
	Let $\mathcal{A}$ be a finitely distributive SLM (distributive CLM).
	Then so is $Lex^k(\mathcal{A})$ %for all $k$ 
	and for any finite subset (any subset) $X \subseteq A^k_L$ 
         we have $(\bigvee X)_1 = \bigvee \{ a \mid \langle a \rangle \in X_{\mid 1}\}$
         and $(\bigvee X)_{i+1} = \bigvee \{ a \mid \langle (\bigvee X)_1, \ldots, (\bigvee X)_i, a \rangle \in X_{\mid i+1}\}$.
\end{theorem}
\begin{proof}
	We know that $Lex^k(\mathcal{A})$ is a finitely distributive SLM (distributive CLM). We just need to check that
	 the characterisation of $\bigvee X$ is correct. So, let with $t^{\mid i}$ its suffix $\langle a_i, \ldots, a_k \rangle$
	 and $Y = \{ t \in X \mid t_1 = (\bigvee X)_1\} \subseteq X$. 
	If $Y= \emptyset$ then $s_1 <^k_L (\bigvee X)_1$ for all $s \in S$ and we are done,
	since $(\bigvee X)^{\mid 2} = \bot^{k-1}_\mathcal{A}$ ensures minimality.
	Otherwise, clearly $\bigvee X = \bigvee Y$ and then we proceed 
	recursively with  $Z =  \{ t \in Y \mid t_2 = (\bigvee Y)_2\} \subseteq Y$,
	since again $Z = \emptyset$ or $\bigvee Y = \bigvee Z$.
\end{proof}

An operational characterisation for residuation can also be obtained. 
First, we need some additional definitions and technical lemmas.

\begin{definition}
	Let $\langle A, \leq, \monop, \odiv, \1 \rangle$
	be a residuated POM with bottom and $s, t \in A^k_L$. Then 
	%		\item $\beta(a) = min\{ i \mid a_i \in C(A)\}$
	we define the partial functions 
	$\gamma(s, t) = \min\{ i \mid (s_i \odiv t_i) \in C(A)\}$
	and $\delta(s, t) = \min\{ i \mid (s_i \odiv t_i) \otimes s_i < t_i\}$.
%	with the convention that the result is $k+1$ whenever the set if empty.
\end{definition}

We adopt the usual convention $\delta(s,t)\downarrow$ and $\delta(s,t)\uparrow$ for 
the partial function $\delta$ being respectively defined or undefined on inputs $s$ and $t$,
and the same for $\gamma$.

%Recall that if $\langle A, \leq, \monop, \1 \rangle$ be a residuated POM, then 
%so is its Cartesian product $A^k$, with division defined point-wise.

\begin{lemma}\label{limit}
	Let $\langle A, \leq, \monop, \odiv, \1 \rangle$
	be a residuated POM with bottom and $s, t \in A^k_L$. Then 
	%either $\delta(a,b) = k+1$ or 
	$\delta(s,t) \leq \gamma(s,t)$
	whenever both 	$\delta(s,t)$ and $\gamma(s,t)$
	are defined.
\end{lemma}
\begin{proof}
	If $\gamma(s,t) < \delta(s,t) \leq k$ then 
	$t_{\gamma(s,t)} \otimes (s_{\gamma(s,t)} \odiv t_{\gamma(s,t)})  = s_{\gamma(s,t)}$.
	Since $C(A)$ is an ideal, it holds that $s_{\gamma(s,t)} \in C(A)$, 
	which in turn implies that $t_{\delta(s,t)} \otimes (s_{\delta(s,t)} \odiv t_{\delta(s,t)}) < s_{\delta(s,t)} = \bot$,
	hence a contradiction. 
\end{proof}

It then suffices to consider only the cases identified by the proposition above, and
$s \cdot t$ denotes string concatenation.
%The key remark for the following propositions is that since in a residuated POM 
%the $\otimes$ operator is monotone, 
%we have that $c \leq a \odiv b$ implies $b \otimes c \leq (a \odiv b) \otimes b$.

\begin{theorem}\label{prop:lexiRes}
	Let $\mathcal{A}$ be a residuated POM with bottom
	element $\bot_\mathcal{A}$.
	Then so is $Lex^k(\mathcal{A})$ and for any pair
	$s , t \in A^k_L$ we have
	%\small
	\[
	s \odiv^k_L t = 	\begin{cases}
	\begin{array}{lcl}
	s \odiv^k_\mathcal{A} t & & \mbox{if  } \gamma(s,t) \uparrow \wedge \delta(s,t) \uparrow \\
	(s \odiv^k_\mathcal{A} t)_{\mid \delta(s,t)} \cdot \top_{Lex^{k-\delta(s,t)}(\mathcal{A})} & &  \mbox{if  } \gamma(s,t) \uparrow \vee \delta(s,t) < \gamma(s,t) \\
	(s \odiv^k_\mathcal{A} t)_{\mid \gamma(s,t)} \cdot \bot^{k - \gamma(s, t)}_\mathcal{A} & & \mbox{if  } \delta(s,t)  \uparrow \vee \delta(s,t) = \gamma(s,t)
	\end{array}
	\end{cases}
	\]
\end{theorem}

\begin{proof}
As for binary products, we need to prove that for all $s \in A^k_L$ we have
$$t \otimes^k_\mathcal{A} z \leq^k_L s \iff z \leq^k_L s \odiv^k_L t$$
%
Recall that $\otimes^k_\mathcal{A}$ works component-wise and 
$s_i \odiv_\mathcal{A} t_i = \bigvee \{s \mid t_i \otimes_\mathcal{A} s \leq_\mathcal{A} s_i\}$.

\begin{itemize}
\item[$\Rightarrow$] Clearly $z_1 \leq_\mathcal{A} s_1 \odiv_\mathcal{A} t_1$ holds.
If $z_1 <_\mathcal{A} s_1 \odiv_\mathcal{A} t_1$ we are done, so let 
$z_1 = s_1 \odiv_\mathcal{A} t_1$: it implies either $\delta(s,t)\uparrow$ or $\delta(s,t) \geq 1$.
If $z_1 \in C(\mathcal{A)}$ then $\delta(s,t) = \gamma(s,t) = 1$ and we are done, 
since it implies $z^{\mid 2} = \bot^{k-1}_\mathcal{A}$, so let $z_1 = s_1 \odiv_\mathcal{A} t_1 \in I(\mathcal{A)}$.
If $\delta(s,t) = 1$ then, by having chosen $\top_{Lex^{k-\delta(s,t)}(\mathcal{A})}$ we are done, so let $\delta(s,t) > 1$.
We can then repeat the argument recursively on $z^{\mid 2}$ and $s^{\mid 2} \odiv^{k-1}_\mathcal{A} t^{\mid 2}$, 
and we are done.

\item[$\Leftarrow$]
Clearly $t_1 \otimes_\mathcal{A} z_1 \leq_\mathcal{A} s_1$ holds.
If $t_1 \otimes_\mathcal{A} z_1 <_\mathcal{A} s_1$ we are done, so let 
$t_1 \otimes_\mathcal{A} z_1 = s_1$: it implies $t_1 \in D(s_1)$ and thus either $\delta(s,t)\uparrow$ or $\delta(s,t) > 1$.
%If $a \in C(\mathcal{A)}$ then 
%$b = \bot_\mathcal{B}$ and we are done, so let $a \in I(\mathcal{A)}$.
If $s_1 \odiv_\mathcal{A} t_1 \in C(A)$ then either $\gamma(s,t) \uparrow$ or $\gamma(s,t)=1$
and we are done, since it implies $\gamma(s,t)\uparrow$ and $\delta(s,t)\uparrow$,
so let $s_1 \odiv_\mathcal{A} t_1 \in I(A)$.
We can then repeat the argument recursively on $t^{\mid 2} \otimes^{k-1} z^{\mid 2}$ and
$s^{\mid 2}$, and we are done,
since $t_1 \in D(s_1)$,
$s_1 \odiv_\mathcal{A} t_1 \in I(A)$ 
and $t^{\mid 2} \otimes^{k-1} z^{\mid 2} \leq^{k-1}_L s^{\mid 2}$
implies 
$t \otimes^k_\mathcal{A} z \leq^k_L s$.
\end{itemize}
\end{proof}

Note that $\top_{Lex^k(\mathcal{A})} = \top^k_\mathcal{A}$
if $\top_\mathcal{A} \in I(A)$, and  $\top_\mathcal{A} \cdot \bot^{k-1}_\mathcal{A}$ 
otherwise.

\subsection{$\omega$-iterated orders}\label{sec:infinite}
We can now move to POMs whose elements are tuples of infinite length.
As for the finite case, for a POM $\mathcal{A}$ we denote by $\mathcal{A}^\omega$
the obvious POM whose elements are tuples of infinite length
of elements in $\mathcal{A}$.

\begin{proposition}
Let $\mathcal{A}$ be a POM with bottom element $\bot_\mathcal{A}$.
%
Then the tuple $Lex^\omega(\mathcal{A}) = \langle A_L^\omega, \leq_L^\omega, \monop^\omega_\mathcal{A}, \1^\omega_\mathcal{A} \rangle$ 
is a POM	with bottom element $\bot^\omega_\mathcal{A}$ such that 
$A_L^\omega = I(A)^\omega \cup I(A)^\ast \times A \times \{\bot_\mathcal{A}\}^\omega$
and $\langle a_1, \ldots, a_k, \ldots \rangle <^\omega_L \langle b_1, \ldots, b_k, \ldots \rangle$ if there exists $j$ such that
$a_i = b_i$ for all $i < j$ and %either $j = k$ and $a_k \leq_\mathcal{A}  b_k$ or 
$a_{j} <_\mathcal{A}  b_{j}$.
\end{proposition}

The result generalises the one in Proposition~\ref{pomLex},
and it clearly holds: transitivity is given by the choice of prefixes in $I(A)$ and suffixes 
in $\{\bot_\mathcal{A}\}^\omega$.

\begin{remark}
	Note that the seemingly obvious POM structure cannot be 
	lifted to $\bigcup_k Lex^k(A) =  I(A)^\ast A \{\bot\}^\ast$: 
	it would be missing the identity of the monoid.
\end{remark}

\begin{proposition}\label{prop:lexiSLM}
	Let $\mathcal{A}$ be a finitely distributive SLM (distributive CLM).
	Then so is $Lex^\omega(\mathcal{A})$
	and for any finite subset (any subset) $X \subseteq A^\omega_L$ 
         we have $(\bigvee X)_1 = \bigvee \{ a \mid \langle a \rangle \in X_{\mid 1}\}$
         and $(\bigvee X)_{i+1} = \bigvee \{ a \mid \langle (\bigvee X)_1, \ldots, (\bigvee X)_i, a \rangle \in X_{\mid i+1}\}$.
\end{proposition}
\begin{proof}
	The proof that $\bigvee X$ of the LUB follows the same steps of the one given in Theorem~\ref{theo:lexiSLM}.
	For distributivity we need to prove that for all $s \in Lex(\mathcal{A})$ and 
	$X \subseteq Lex^\omega(\mathcal{A})$ we have $s \otimes^\omega_\mathcal{A} \bigvee X = \bigvee s \otimes^\omega_\mathcal{A} X$, 
	with $s \otimes^\omega_\mathcal{A} X = \{ s \otimes^\omega_\mathcal{A} t \mid t \in X\}$.
	
	We retrace the steps of Proposition~\ref{def:lexBI}, so let $Y = \{ s \otimes^\omega_\mathcal{A} t \mid t \in X\}$..
	We have that 
	$(s \otimes^\omega_\mathcal{A} \bigvee X)_1 = s_1 \otimes_\mathcal{A} (\bigvee X)_1 = s_1 \otimes_\mathcal{A} \bigvee X_1
	= \bigvee s_1 \otimes X_1 =  \bigvee Y_1 = (\bigvee Y)_1$
	by distributivity of $\otimes_\mathcal{A}$.
	%
	If $s_1 \in C(A)$ then $Y_1 \subseteq C(\mathcal{A})$ and thus $Y_2 = \{\bot_\mathcal{A}\}$ and we are done, since 
	$s \otimes^\omega \bigvee X = (s \otimes_\mathcal{A} \bigvee X)_{\mid 1} \cdot \bot^\omega_\mathcal{A}
	= (\bigvee Y)_{\mid 1} \cdot  \bot^\omega_\mathcal{A}  = \bigvee Y$.
	%
	Otherwise, 
	let $X2 = \{ t \mid (\bigvee X)_{\mid 1} \cdot t \in X\}$ and note that 
	$$Y2 = \{ w \mid (\bigvee Y)_{\mid 1} \cdot w \in Y\}
	= \{ w \mid (s \otimes^\omega_\mathcal{A} \bigvee X)_{\mid 1} \cdot w \in Y \}$$
	\noindent
	Now, note that $(\bigvee X)_1 \not \in X_1$ implies $X2 = \emptyset$ as well as, since $s_1$ is cancellative, $Y2 = \emptyset$.
	If $X2 = Y2 = \emptyset$ we are done, so let us assume that $(\bigvee X)_1 \in X_1$. Then
%	$$\{ b \otimes_\mathcal{B} z \mid z \in X'_\mathcal{B} \wedge 
%	      \langle a \otimes_\mathcal{A} \bigvee X_\mathcal{A}, b  \otimes_\mathcal{B} z \rangle \in \langle a, b\rangle \otimes_{\mathcal{A} \times \mathcal{B}} X \rangle \} =$$
	$$Y2 = \{ s^{\mid 2} \otimes^\omega_\mathcal{A}  z \mid (\bigvee X)_1 \in X_1 \wedge z \in X2 \wedge 
	       (s \otimes^\omega_\mathcal{A} \bigvee X)_{\mid 1} \cdot (s^{\mid 2} \otimes^\omega_\mathcal{A}  z) \in Y \} =$$
	$$\{ s^{\mid 2} \otimes^\omega_\mathcal{A} z \mid  (\bigvee X)_1 \in X_1 \wedge z \in X2\} =
	    s^{\mid 2} \otimes^\omega_\mathcal{A} \{ z \mid (\bigvee X)_{\mid 1} \cdot z \in X \} = s^{\mid 2} \otimes^\omega_\mathcal{A} X2$$
	\noindent
	and then we proceed recursively to prove that $\bigvee Y2 = \bigvee s^{\mid 2} \otimes^\omega_\mathcal{A} X2$.
\end{proof}

%Next result is instead an adaptation of Theorem~\ref{prop:lexiRes}

\begin{proposition}\label{prop:lexiResOmega}
	Let $\mathcal{A}$ be a residuated POM with bottom $\bot_\mathcal{A}$.
	Then so is $Lex^\omega(\mathcal{A})$
	and for any pair $s, t \in A^\omega_L$ we have
	\[
	s \odiv^\omega_L t = 	\begin{cases}
	\begin{array}{lcl}
	s \odiv^\omega_\mathcal{A} t & & \mbox{if  } \gamma(s,t) \uparrow \wedge \delta(s,t) \uparrow \\
	(s \odiv^\omega_\mathcal{A} t)_{\mid \delta(s,t)} \cdot \top_{Lex^\omega(\mathcal{A})} & &  \mbox{if  } \gamma(s,t) \uparrow \vee \delta(s,t) < \gamma(s,t) \\
	(s \odiv^\omega_\mathcal{A} t)_{\mid \gamma(s,t)} \cdot \bot^\omega_\mathcal{A} & & \mbox{if  } \delta(s,t)  \uparrow \vee \delta(s,t) = \gamma(s,t)
	\end{array}
	\end{cases}
	\]
\end{proposition}

The proof is an adaptation of Theorem~\ref{prop:lexiRes}, via the obvious extension of Lemma~\ref{limit}.
Note that the top is $\top_\mathcal{A}^\omega$
if $\top_\mathcal{A} \in I(\mathcal{A})$, and  $\langle \top_\mathcal{A} \rangle \cdot \bot_\mathcal{A}^\omega$ otherwise.

\section{Conclusions}\label{sec:conclusion}
In this paper we considered a framework for soft constraint formalisms based on a residuated monoid of partially ordered preferences.  
%
More specifically, our focus was to show that the framework can include also lexicographic orders.

Despite the practical relevance of these orders, the valued structures for constraints proposed in the literature, 
such as the pioneering~\cite{jacm97,schiex}, cannot straightforwardly deal 
with such a class of preferences, and their modelling required some ingenuity~\cite{GadducciHMW13,valuation}. 
%
Generalising such proposals by using residuated monoids~\cite{residuation1} allows the extension of the classical solving algorithms 
that need preference removal, such as arc consistency, where values need to be moved from binary to unary constraints~\cite{ipl}.
%
The paper shows that the resolution techniques devised for this richer framework can tackle also lexicographic orders.

Our current work is on the design of specific heuristics to specialise such techniques to constraint satisfaction problems involving these orders. 

% (see Section~\ref{sec:bucket}). 
%
%The contribution of this paper is twofold. On the one side, we proved the adequacy of the formalism for modelling lexicographic orders. 
%On the other side, we showed how it can enable heuristics for efficiently solving soft CSPs, such as the Nucket and Mini-bucket elimination.
%
%Notice that the work in \cite{bucketsemiring} uses  an $\odiv$ operator that cannot be used on lexicographic orders. 
%Moreover, having provided a definition of residuation on infinite tuples enables the use of residuation in temporal reasoning formalisms 
%(e.g., Allen's interval algebra and its fragments)~\cite{infinite} with lexicographic orders, as for example accomplished with 
%soft constraint automata in \cite{sca}.

%Our focus on soft CSP includes its computational counterparts based on constraints, such as soft CCP~\cite{labelled}, and in fact, considering infinite tuples 
%enables to model temporal reasoning as shown for soft constraint automata in \cite{sca}. However, the framework is reminiscent, and is in fact an extension, 
%of previous formalisms such as monotonic logic programming~\cite{residuatedlogic}, whose semantics is given in terms of residuated 
%lattices of complete lattices of truth-values. And it fits in the current interests on the development of sequent systems for substructural logics, 
%as witnessed by current research projects~\cite{ciabattoni}:
%
%where $\otimes$ and $\leftarrow$ (implication) form an adjoint pair. Possibilistic 
%values are then represented through a residuated algebra on $[0, 1]$.
%Residuated lattices can be used as algebraic structures for \emph{substructural logics}~\cite{ciabattoni} 
%when they are formalised in sequent systems, thus lacking some or all of the structural rules: 
%well-known examples are Lukasiewicz's many-valued logics, relevance logics and linear logics.
%As can be The ultimate goal is often to generalise logic program results to richer algebraic structures.

\bibliographystyle{elsarticle-num}
\bibliography{main}
\end{document}

\section{Roba da riciclo}

\begin{proposition}
	\label{div0}
	Let $\langle A, \leq, \monop, \odiv, \1 \rangle$
	be a residuated POM with bottom and $a, b \in Lex_k(A)$.
	If $\delta(a,b) = \gamma(a,b) = k+1$
	then their residuation $a \odiv_k b$ 
	in $Lex_k(A)$ exists and it is given by 
	%$a \odiv^k b$, that is
	$(a_1 \odiv b_1) \ldots  (a_k \odiv b_k)$.
\end{proposition}

\begin{proof}
	First of all, note that 
	$a \odiv_k b \in I(A)^k \subseteq Lex_k(A)$.
	% and recall that 
	%$\forall i < \delta(a,b).\ a_i = (a_i \odiv b_i) \otimes b_i$.
	%
	So, given $c \in Lex_k(A)$, we need to prove that
	$b\otimes^k c \leq_k a$ iff $c \leq_k a \odiv_k b$.
	
	\begin{description}
		\item[$\mathbf{[b\otimes^k c \leq_k a].}$]
		Let $l = \min \{ i \mid b_i \otimes c_i < a_i\}$,
		with the convention that the result is $k+1$ whenever the set if empty.
		Also, let $m = \min\{l, k\} < \delta(a,b)$.
		
		We have that $b_j \otimes c_j \leq a_j$ for all $j \leq m$, hence
		$c_j \leq a_j \odiv b_j $ for all $j \leq m$. 
		%
		If $c_n < a_n \odiv b_n$ for some $n \leq m$
		we are done. 
		%
		Otherwise, if $m = k < l$ then 
		$c_n = a_n \odiv b_n$ for all $n \leq k$ and we are done.
		%
		Finally, if $m = l \leq k $ then 
		$b_l \otimes c_l = b_l  \otimes (a_l \odiv b_l) = a_l$
		since $l < \delta(a,b)$, a contradiction.
		
		\item[$\mathbf{[c \leq_k a \odiv_k b].}$]
		Let $l = \min \{ i \mid c_i < (a \odiv_k b)_i\}$,
		with the convention that the result is $k+1$ whenever the set if empty.
		Also, let $m = \min \{l, k\} < \delta(a,b)$.
		
		We have that $c_j \leq (a \odiv_k b)_j = a_j \odiv b_j$ for all $j \leq m$, hence
		$b_j \otimes c_j \leq a_j$ for all $j \leq m$.
		%
		If $b_n \otimes c_n < a_n$ for some $n \leq m$
		we are done. 
		%
		Otherwise, if $m = k < l$ then 
		$b_n \otimes c_n = a_n$ for all $n \leq k$	and we are done.
		%
		Finally, if $m = l \leq k$ then 
		$b_l \otimes c_l = a_l = b_l \otimes (a_l \odiv b_l)$
		since $l < \delta(a,b)$, hence either a contradiction
		if $b_l \in I(A)$ or $b_p \otimes c_p = \bot = a_p$ for all $p > l$
		if $b_l \in C(A)$ and consequently $a_l \in C(A)$. \qedhere
	\end{description}
\end{proof}

Note that $a \odiv_k b$ here coincides with the residuation $a \odiv^k b$ 
on the Cartesian product. Furthermore, we have 
that $(a \odiv_k b) \otimes^k b = a$.

\begin{proposition}
	\label{div1}
	Let $\langle A, \leq, \monop, \odiv, \1 \rangle$
	be a residuated POM with bottom and $a, b \in Lex_k(A)$.
	If $\delta(a,b) < \gamma(a,b)$ then their residuation $a \odiv_k b$ 
	in $Lex_k(A)$ exists and it is given by
	%	\[ \left\{
	%	   \begin{array}{l}
	%         	     (a_1 \odiv b_1) \ldots  (a_{\delta(a, b)}  \odiv b_{\delta(a, b)}) 
	%	              \top^{k - \delta(a, b)} \mbox{  if  }  \top \in I(A) \\
	%           	     (a_1 \odiv b_1) \ldots  (a_{\delta(a, b)}  \odiv b_{\delta(a, b)}) 
	%	             \top \bot^{k - \delta(a, b) - 1} \mbox{  otherwise}
	%	    \end{array}
	%	\right
	%	\]
	%oppure
$
	(a_1 \odiv b_1) \ldots  (a_{\delta(a, b)}  \odiv b_{\delta(a, b)}) 
	(\bigvee Lex_{k - \delta(a, b)}(A))
$.
	%	such that $x = \top$ if $\top \in I(A)$, and  $x = \bot$ otherwise.
\end{proposition}

\begin{proof}
	First of all, note that 
	$(a \odiv_k b)_{\mid \delta(a,b)} \in I(A)^{\delta(a,b)} \subseteq Lex_{\delta(a,b)}(A)$.
	% and recall that 
	%$\forall i < \delta(a,b).\ a_i = (a_i \odiv b_i) \otimes b_i$.
	%
	So, given $c \in Lex_k(A)$, we need to prove that
	$b\otimes^k c \leq_k a$ iff $c \leq_k a \odiv_k b$.
	
	\begin{description}
		\item[$\mathbf{[b\otimes^k c \leq_k a].}$]
		Let $l = \min \{ i \mid b_i \otimes c_i < a_i\}$,
		with the convention that the result is $k+1$ whenever the set if empty.
		Also, let $m = \min \{l, \delta(a,b)\}$.
		
		We have that $b_j \otimes c_j \leq a_j$ for all $j \leq m$, hence
		$c_j \leq a_j \odiv b_j $ for all $j \leq m$. 
		%
		If $c_n < a_n \odiv b_n$ for some $n \leq m$
		we are done. 
		%
		Otherwise, if $m = \delta(a,b) \leq l$ then 
		$c_j \leq a_j \odiv b_j $ for all $j \leq \delta(a,b)$ and we are done.
		%
		Finally, if $m = l < \delta(a,b)$ then 
		$b_l \otimes c_l = b_l \otimes (a_l \odiv b_l) = a_l$,
		 a contradiction.
		
		\item[$\mathbf{[c \leq_k a \odiv_k b].}$]
		Let $l = \min \{ i \mid c_i < (a \odiv_k b)_i\}$,
		with the convention that the result is $k+1$ whenever the set if empty.
		Also, let $m = \min \{l, \delta(a,b)\}$.
		
		We have that $c_j \leq (a \odiv_k b)_j = a_j \odiv b_j$ for all $j \leq m$, hence
		$b_j \otimes c_j \leq a_j$ and $b_j \otimes c_j \leq b_j \otimes (a_j \odiv b_j)$ 
		for all $j \leq m$, the latter by monotonicity of $\otimes$.
		%
		If $b_n \otimes c_n < a_n$ for some $n \leq m$
		we are done. 
		%
		Otherwise, if $m = \delta(a,b) \leq l$ then 
		$b_{\delta(a,b)} \otimes (a_{\delta(a,b)} \odiv b_{\delta(a,b)}) <
		a_{\delta(a,b)} = b_{\delta(a,b)} \otimes c_{\delta(a,b)} \leq
		b_{\delta(a,b)} \otimes (a_{\delta(a,b)} \odiv b_{\delta(a,b)})$,
		hence a contradiction.
		%
		Finally, if $m = l < \delta(a,b)$ then 
		$b_l \otimes c_l = a_l = b_l \otimes (a_l \odiv b_l)$, 
		hence either a contradiction
		if $b_l \in I(A)$ or $b_p \otimes c_p = \bot = a_p$ for all $p > l$
		if $b_l \in C(A)$ and consequently $a_l \in C(A)$. \qedhere
	\end{description}
\end{proof}

Additionally, please note that $\bigvee Lex_n(A)$ can be easily characterised: 
it coincides with $\top^n$
if $\top \in I(A)$, and with $\top \bot^{n -1}$ otherwise.

\begin{proposition}
	\label{div2}
	Let $\langle A, \leq, \monop, \odiv, \1 \rangle$
	be a residuated POM with bottom element $\bot$ and $a, b \in Lex_k(A)$.
	If $\delta(a,b) = \gamma(a,b) \leq k$ or $ \gamma(a,b) < \delta(a,b)$
	then their residuation $a \odiv_k b$ 
	in $Lex_k(A)$ exists: it is given by
	$(a_1 \odiv b_1) \ldots  (a_{\gamma(a, b)}  \odiv b_{\gamma(a, b)}) 
	\bot^{k - \gamma(a, b)}$.
\end{proposition}

\begin{proof}
	First of all, note that 
	$(a \odiv_k b)_{\mid \gamma(a,b)} \in I(A)^{\gamma(a,b)-1} C(A) \subseteq Lex_{\gamma(a,b)}(A)$.
	Also, $\delta(a,b) >  \gamma(a,b)$ implies that 
	$\delta(a,b) = k+1$ and $a_{\gamma(a,b)} \in C(A)$.
	% and recall that 
	%$\forall i < \delta(a,b).\ a_i = (a_i \odiv b_i) \otimes b_i$.
	%
	Given $c \in Lex_k(A)$, we need to prove that
	$b\otimes^k c \leq_k a$ iff $c \leq_k a \odiv_k b$.
	
	\begin{description}
		\item[$\mathbf{[b\otimes^k c \leq_k a].}$]
		Let $l = \min \{ i \mid b_i \otimes c_i < a_i\}$,
		with the convention that the result is $k+1$ whenever the set if empty.
		Also, let $m = \min \{l, \gamma(a,b)\} \leq \delta(a,b)$.
		
		We have that $b_j \otimes c_j \leq a_j$ for all $j \leq m$, hence
		$c_j \leq a_j \odiv b_j $ for all $j \leq m$. 
		%
		If $c_n < a_n \odiv b_n$ for some $n \leq m$
		we are done. 
		%
		Otherwise, if $m = \gamma(a,b) \leq l$ then 
		$c_{\gamma(a,b)} \in C(A)$ and we are done since
		$c_p = \bot = (a \odiv_k b)_p$ for all $p > {\gamma(a,b)}$.
		%
		Finally, if $m = l < \gamma(a,b)$ then 
		$b_l \otimes c_l = b_l \otimes (a_l \odiv b_l) = a_l$
		since $l < \delta(a,b)$, a contradiction.
		
		\item[$\mathbf{[c \leq_k a \odiv_k b].}$]
		Let $l = \min \{ i \mid c_i < (a \odiv_k b)_i\}$,
		with the convention that the result is $k+1$ whenever the set if empty.
		Also, let $m = \min \{l, \gamma(a,b)\} \leq \delta(a,b)$.
		
		We have that $c_j \leq (a \odiv_k b)_j = a_j \odiv b_j$ for all $j \leq m$, hence
		$b_j \otimes c_j \leq a_j$ and $b_j \otimes c_j \leq b_j \otimes (a_j \odiv b_j)$ 
		for all $j \leq m$, the latter by monotonicity of $\otimes$.
		%
		If $b_n \otimes c_n < a_n$ for some $n \leq m$
		we are done. 
		%
		Otherwise, if $m = \gamma(a,b) < l$ then 
		$c_{\gamma(a,b)} \in C(A)$ and we are done since
		$b_p \otimes c_p = \bot \leq a_p$ for all $p > {\gamma(a,b)}$.
		%
		Finally, if $m = l \leq \gamma(a,b)$ then 
		$b_l \otimes (a_l \odiv b_l) \leq a_l  = b_l \otimes c_l \leq b_l \otimes (a_l \odiv b_l)$
		since $l \leq \delta(a,b)$, hence either a contradiction
		if $b_l \in I(A)$ or $b_p \otimes c_p = \bot = a_p$ for all $p > l$
		if $b_l \in C(A)$ and consequently $a_l \in C(A)$. \qedhere
	\end{description}
\end{proof}


%Note that when $\delta(a,b) = \infty$ we have $\beta(a) \leq \gamma(a,b)$.
%
%\begin{remark}
%	In fact, we can be a bit more specific.
%	Since $\delta(a,b) = \infty$ we know that either $b_{\beta(a)} \in C(A)$ or 
%	$a_{\beta(a)} \odiv b_{\beta(a)} \in C(A)$.
%	In the latter case $\beta(a) = \gamma(a,b)$.
%	If the latter case is false and the former one holds, then
%	$a_{\beta(a)+1} \odiv b_{\beta(a)+1} = \bot \odiv \bot = \top$,
%	so either $\gamma(a,b) = \beta(a)+1$ if  $\top \in C(A)$ or 
%	$\gamma(a,b) = \infty$ otherwise.
%\end{remark}

From the propositions above it is straightforward to derive Theorem~\ref{prop:lexiRes}, 
which states that, given a residuated POM, it is possible to define a lexicographic 
order on its tuples, which is a residuated POM as well. 

\begin{theorem}\label{prop:lexiRes}
	Let $\mathcal{A}$ be a residuated POM with bottom
	element $\bot_\mathcal{A}$.
	Then so is $Lex^k(\mathcal{A})$ and for any pair
	$s , t \in A^k_L$ we have
	%\small
	\[
	s \odiv^k_L t = 	\begin{cases}
	\begin{array}{lcl}
	s \odiv^k t & & \mbox{if  } \gamma(s,t) \uparrow \wedge \delta(s,t) \uparrow \\
	(s_1 \odiv t_1) \ldots  (s_{\delta(s, t)}  \odiv t_{\delta(s, t)})
	\top_{Lex^{k-\delta(a,b)}(\mathcal{A})} & &  \mbox{if  } \gamma(a,b) \uparrow \vee \delta(a,b) < \gamma(s,t) \\
	(a_1 \odiv b_1) \ldots  (a_{\gamma(a, b)}  \odiv b_{\gamma(a, b)}) 	
	\bot^{k - \gamma(a, b)} & & \mbox{if  } \gamma(a,b) \mbox { defined}, \delta(a,b) \mbox { undefined}
	\end{array}
	% versione compatta basata sui cartesiani \odiv^\omega
	%	                                \begin{array}{lcl}
	%        		                           a \odiv^k b & & \mbox{if  } k+1 = \gamma(a,b) = \delta(a,b) \\
	%                                           (a \odiv^k b)_{\mid \gamma(a, b)}  	
	%	                                   \bot^{k - \gamma(a, b)} & & \mbox{if  } k+1 \neq \gamma(a,b) \leq \delta(a,b) \\
	%           	                         (a \odiv^k b)_{\mid \delta(a, b)}
	%	                                  (\bigvee Lex_{k-\delta(a,b)}(A)) & & \mbox{otherwise}
	%	                                 \end{array}
	\end{cases}
	\]
\end{theorem}
%\begin{itemize}
%\item $(a \odiv_k b)_1 = a_1 \odiv b_1$
%\item $(a \odiv_k b)_{i+1} = a_{i+1} \odiv b_{i+1}$ if $(a \odiv_k b)_i \in I(A)$
%and $\bot$ otherwise
%\end{itemize}
%\end{proposition}
%\begin{proof}
%A straightforward consequence of Prop.~\ref{div1} and Prop.~\ref{div2}.
%\end{proof}

%\begin{remark}
%Is the proposition true? Also, its proof above would be simplified should $I(A)$ 
%be closed with respect to residuation.
%\end{remark}
\end{document}
